\documentclass[german]{article}

\usepackage{german}
\usepackage[utf8]{inputenc}
\usepackage{enumerate}
\usepackage{enumitem}

\setlength{\textwidth}{15cm}
\setlength{\textheight}{24cm}
\addtolength{\topmargin}{-2cm}
\addtolength{\oddsidemargin}{-1.5cm}

\title{\textsf{\textbf{Satzung des TechAachen e.V.}}}
\author{}
\date{}

\setlist[enumerate,1]{label=§\,\arabic*,ref=§\,\arabic*}
\setlist[enumerate,2]{label=\arabic*.,ref=\theenumi.\arabic*}
\setlist[enumerate,3]{label=(\alph*),ref=\theenumii\,(\alph*)}

\newcommand{\paragr}[1]{\item \textsf{\textbf{#1}}}


\begin{document}
\maketitle

\begin{enumerate}
\paragr{Name und Sitz}

\begin{enumerate}

\item Der Verein führt den Namen TechAachen.

\item Er soll in das Vereinsregister eingetragen werden und trägt dann den Zusatz ``e.V.''

\item Der Sitz des Vereins ist Aachen.

\end{enumerate}

\paragr{Geschäftsjahr}

Geschäftsjahr ist das Kalenderjahr.


\paragr{Zweck des Vereins}
\begin{enumerate}

\item Der Verein verfolgt ausschließlich und unmittelbar gemeinnützige Zwecke im Sinne des Abschnitts ``Steuerbegünstigte Zwecke'' der Abgabenordnung.

\item Zweck des Vereins ist \label{zweck}
\begin{enumerate}
\item \label{zweck_wissenschaft_forschung} die Förderung von Wissenschaft und Forschung;
\item \label{zweck_bildung} die Förderung der Erziehung, Volks- und Berufsbildung einschließlich der Studentenhilfe;
\end{enumerate}
weiterhin ist der Verein eine Mittelbeschaffungskörperschaft nach § 58 AO.

\item Der Satzungszweck wird verwirklicht insbesondere durch
\begin{enumerate}
\item Förderung der Zusammenarbeit regional ansässiger Organisationen, die im Sinne der Vereinszwecke tätig sind, u.a. die sog. technischen studentischen Eigeninitiativen der RWTH Aachen und der FH Aachen;
\item die Beschaffung von Mitteln und deren Weiterleitung zur Förderung der steuerbegünstigten Zwecke im Sinne der Absätze \ref{zweck_wissenschaft_forschung} und \ref{zweck_bildung};
\item Förderung des Dialogs zwischen Organisationen und der Öffentlichkeit, der Politik, der Wirtschaft, den Hochschulen und Forschungseinrichtungen in der Region;
\item Durchführung und Förderung von Informationsveranstaltungen und Bildungsveranstaltungen, insbesondere in den Bereichen Ingenieurwesen, Marketing, Recht, Buchführung.
\item Schaffung einer Plattform für das Umsetzen gemeinnütziger Projekte im Sinne der Absätze \ref{zweck_wissenschaft_forschung} und \ref{zweck_bildung}. Diese sind insbesondere Entwicklungen in den Bereichen Ingenieurwesen, Marketing, Recht, Buchführung.
\end{enumerate}

\item Der Verein ist parteipolitisch und religiös neutral.

\end{enumerate}

\paragr{Selbstlose Tätigkeit}

Der Verein ist selbstlos tätig; er verfolgt nicht in erster Linie eigenwirtschaftliche Zwecke.


\paragr{Mittelverwendung}

Mittel des Vereins dürfen nur für die satzungsmäßigen Zwecke verwendet werden.


\paragr{Verbot von Begünstigungen}

Es darf keine Person durch Ausgaben, die dem Zweck der Körperschaft fremd sind, oder durch unverhältnismäßig hohe Vergütungen begünstigt werden.


\paragr{Mitgliedschaft}
\label{mitglieder}

Es gibt drei Arten von Mitgliedern:

\begin{enumerate}

\item Körperschaften, die einen gemeinnützigen Zweck gemäß §\,52\,AO verfolgen; Diese Mitglieder werden reguläre Mitglieder genannt und haben ein Stimmrecht. \label{mitglieder_regulaer}

\item Natürliche oder juristische Personen als Ehrenmitglieder. Ehrenmitglieder haben kein Stimmrecht auf der Mitgliederversammlung.

\item Natürliche oder juristische Personen als Fördermitglieder. Fördermitglieder haben kein Stimmrecht auf der Mitgliederversammlung.
\end{enumerate}


\paragr{Erwerb der Mitgliedschaft}
\begin{enumerate}

\item Über die Aufnahme eines ordentlichen Mitgliedes entscheidet die Mitgliederversammlung mit Zweidrittelmehrheit.

\item Über die Aufnahme von Fördermitgliedern entscheidet der Vorstand.

\item Ehrenmitglieder werden durch einstimmigen Beschluss der Mitgliederversammlung ernannt.

\item Der Antrag auf Fördermitgliedschaft darf durch den Vorstand unbegründet abgelehnt werden.
Die Mitglieder müssen innerhalb von 14 Tagen über den Antrag informiert werden.

\end{enumerate}


\paragr{Beendigung der Mitgliedschaft}
\begin{enumerate}

\item Die Mitgliedschaft endet durch Austritt, Ausschluss, Tod oder Auflösung der juristischen Person.

\item Der Austritt erfolgt durch schriftliche Erklärung gegenüber dem Vorstand.
Die schriftliche Austrittserklärung muss mit einer Frist von drei Monaten jeweils zum Ende des Geschäftsjahres gegenüber dem Vorstand erklärt werden.

\item Ein außerordentlicher Austritt ist jederzeit aufgrund einer Satzungsänderung oder einer Mehrbelastung durch eine geänderte Beitragsordnung möglich.
Der Austritt gilt zum Zeitpunkt des Inkrafttretens der jeweiligen Änderung und ist spätestens bis 3 Monate nach Bekanntgabe der Änderung einzureichen.

\item Ein Ausschluss kann nur aus wichtigem Grund erfolgen.
Wichtige Gründe sind insbesondere
\begin{enumerate}
\item ein die Vereinsziele schädigendes Verhalten,
\item die Verletzung satzungsmäßiger Pflichten oder
\item Beitragsrückstände von mindestens 6 Monaten oder 3 Mitgliedsbeiträge
\end{enumerate}

\item Über den Ausschluss entscheidet der Vorstand.

\item Gegen den Ausschluss steht dem Mitglied die Berufung an die Mitgliederversammlung zu, die schriftlich binnen eines Monats an den Vorstand zu richten ist.

\item Die Mitgliederversammlung entscheidet im Rahmen des Vereins endgültig.
Dem Mitglied bleibt die Überprüfung der Maßnahme durch Anrufung der ordentlichen Gerichte vorbehalten.

\item Die Anrufung eines ordentlichen Gerichts hat aufschiebende Wirkung bis zur Rechtskraft der gerichtlichen Entscheidung.

\item Das ausscheidende Mitglied verliert jeden Anspruch auf das Vermögen des Vereins.

\item Die bis zum Austritt entstandenen Verpflichtungen gegenüber dem Verein bleiben bestehen.

\item Die Mitglieder dürfen bei ihrem Ausscheiden nicht mehr als ihre eingezahlten Kapitalanteile und den gemeinen Wert ihrer geleisteten Sacheinlagen zurückerhalten.
Über Rückzahlung oder Rückgabe der eingezahlten Kapitalanteile oder geleisteten Sacheinlagen  entscheidet die Mitgliederversammlung.

\end{enumerate}


\paragr{Beiträge}
\begin{enumerate}

\item Von regulären Mitgliedern werden keine Beiträge erhoben.

\item Von Fördermitgliedern können Beiträge erhoben werden.

\item Näheres regelt die Beitragsordnung, diese wird von der Mitgliederversammlung festgelegt.

\end{enumerate}


\paragr{Organe des Vereins}

Organe des Vereins sind
\begin{enumerate}
\item Die Mitgliederversammlung
\item Der Vorstand
\item Die Teams
\item Der Beirat
\end{enumerate}


\paragr{Mitgliederversammlung}
\begin{enumerate}

\item Die Mitgliederversammlung ist das oberste Vereinsorgan.

\item Zu ihren Aufgaben gehören insbesondere
\begin{enumerate}
\item die Wahl und Abwahl des Vorstands,
\item Entlastung des Vorstands,
\item Entgegennahme der Berichte des Vorstandes,
\item Wahl der Kassenprüfer/innen,
\item Festsetzung von Beiträgen und deren Fälligkeit in der Beitragsordnung,
\item Beschlussfassung über die Änderung der Satzung,
\item Beschlussfassung über die Auflösung des Vereins,
\item Entscheidung über Aufnahme und Ausschluss von Mitgliedern in Berufungsfällen,
\item Berufung von Teams,
\item Festsetzung einer Geschäftsordnung der Mitgliederversammlung
\item Festsetzung einer Geschäftsordnung des Vorstandes
\item Festsetzung einer Geschäftsordnung des Beirats
\item Festsetzung einer Finanzordnung
\item Festsetzung einer Nutzungsordnung von Vereinsbesitz
\item sowie weitere Aufgaben, soweit sich diese aus der Satzung oder nach dem Gesetz ergeben.
\end{enumerate}
\item Im erstem Quartal eines jeden Geschäftsjahres findet eine ordentliche Mitgliederversammlung statt.

\item Der Vorstand ist zur Einberufung einer außerordentlichen Mitgliederversammlung verpflichtet, wenn mindestens ein Drittel der Mitglieder dies schriftlich oder per E-Mail unter Angabe von Gründen verlangt.

\item Die Mitgliederversammlung wird vom Vorstand unter Einhaltung einer Frist von zwei Wochen schriftlich, oder per E-Mail unter Angabe der Tagesordnung einberufen.

\item Die Frist beginnt mit dem auf die Absendung des Einladungsschreibens oder der Mail folgenden Tag.

\item Das Einladungsschreiben gilt als den Mitgliedern zugegangen, wenn es an die letzte dem Verein bekannt gegebene Anschrift oder E-Mail-Adresse gerichtet war.

\item Soll der Vorstand entlastet werden ist der Einladung ein Kassenprüfbericht der Kassenprüfer/innen beizulegen.

\item Mit der Einladung zur Mitgliederversammlung ist den Mitgliedern ein Rechenschaftsbericht durch den Vorstand vorzulegen.
Der Rechenschaftsbericht umfasst alle wesentlichen Tätigkeiten des Verein und des Vorstands, insbesondere:
\begin{enumerate}
\item die Verwendung des Vereinsvermögens
\item die Aufnahme von Fördermitgliedern
\item die Erfüllung des Vereinszwecks gemäß \ref{zweck} im Geschäftsjahr
\end{enumerate}

\item Die Tagesordnung ist zu ergänzen, wenn dies ein Mitglied bis spätestens eine Woche vor dem angesetzten Termin schriftlich oder per E-Mail beantragt.
Die Ergänzung ist zu Beginn der Versammlung bekanntzumachen.

\item Anträge
\begin{enumerate}
\item über die Abwahl des Vorstands
\item über die Änderung der Satzung
\item über die Auflösung des Vereins
\item über die Änderung der Beitragsordnung
\item über die Aufnahme von Mitgliedern
\end{enumerate}
die den Mitgliedern nicht bereits mit der Einladung zur Mitgliederversammlung zugegangen sind, können erst auf der nächsten Mitgliederversammlung beschlossen werden.

\item Die Mitgliederversammlung ist ohne Rücksicht auf die Zahl der erschienenen Mitglieder beschlussfähig.

\item Die Mitgliederversammlung wird von einem Vorstandsmitglied geleitet.

\item Zu Beginn der Mitgliederversammlung ist eine anwesende natürliche Person zum Schriftführer zu wählen.

\item Jedes Mitglied hat eine Stimme. Das Stimmrecht kann nur persönlich, fernmündlich, oder für ein Mitglied unter Vorlage einer schriftlichen Vollmacht ausgeübt werden.

\item Bei Abstimmungen entscheidet die einfache Mehrheit der abgegebenen Stimmen.

\item Satzungsänderungen und die Auflösung des Vereins können nur mit einer Mehrheit von 2/3 der anwesenden Mitglieder beschlossen werden.

\item Stimmenthaltungen und ungültige Stimmen bleiben außer Betracht.

\item Über die Beschlüsse der Mitgliederversammlung ist ein Protokoll anzufertigen, das vom Versammlungsleiter und dem Schriftführer zu unterzeichnen ist.

\item Das Protokoll ist innerhalb von 14 Tagen schriftlich oder per E-Mail an die Mitglieder zu versenden.

\item Auf der Mitgliederversammlung wird eine natürliche und anwesende Person zum Wahlleiter gewählt.

\item Näheres regelt die Geschäftsordnung der Mitgliederversammlung.

\end{enumerate}


\paragr{Vorstand}
\begin{enumerate}

\item Der Vorstand im Sinn des §\,26\,BGB besteht aus
\begin{enumerate}
\item dem/der 1., 2. und 3. Vorsitzenden
\end{enumerate}

\item Sie vertreten den Verein gerichtlich und außergerichtlich.
Zwei Vorstandsmitglieder vertreten gemeinsam. Der geschäftsführende Vorstand ist befugt Vorstandsmitglieder in bestimmten Bereichen Einzelvertretungsbefugnis zu erteilen.

\item Der Vorstand wird von der Mitgliederversammlung auf die Dauer von einem Jahr gewählt. 
Diese Posten werden einzeln gewählt. Auf Antrag ist es möglich, in cumulo (Blockwahl) zu wählen.

\item Vorstandsmitglieder sind natürliche Personen.

\item Vorstandsmitglieder müssen durch ein Mitglied auf der Mitgliederversammlung zur Wahl vorgeschlagen werden.

\item Der Vorstand kann außerhalb der regulären Wahlen nur aus wichtigem Grund abgewählt werden und durch einen Neuen ersetzt werden. Wichtige Gründe sind insbesondere:
\begin{enumerate}
\item ein die Vereinsziele schädigendes Verhalten
\item die Verletzung satzungsmäßiger Pflichten
\item Rücktritt vom Amt
\item Geschäftsunfähigkeit durch Krankheit oder Unauffindbarkeit.
\end{enumerate}
\item Wiederwahl ist zulässig.

\item Der Vorstand bleibt solange im Amt, bis ein neuer Vorstand gewählt ist.

\item Dem Vorstand obliegt die Führung der laufenden Geschäfte des Vereins. Er hat insbesondere folgende Aufgaben:
\begin{enumerate}
\item die Führung der Geschäfte des Vereins,
\item die Einberufung und Vorbereitung der Mitgliederversammlung,
\item die Ausführung von Beschlüssen der Mitgliederversammlung,
\item die Verwaltung über das Vereinsvermögens,
\item die Anfertigung eines schriftlichen Jahresberichts,
\item die Berufung von Teams.
\end{enumerate}
Bei alle Aufgaben und Verpflichtungen hat der Vorstand die Entscheidungsbefugnis bis einschließlich 5000,- Euro. Darüberhinausgehende Entscheidungen und Verpflichtungen unterliegen der Beschlussfassung der geschäftsführenden Versammlung.

\item Die Beschlüsse des Vorstandes sind zu protokollieren und die Vereinsmitglieder über diese zu informieren.

\item Der Vorstand kann zur Erledigung der laufenden Geschäfte eine oder mehrere Personen berufen, diese werden im Rahmen seiner Weisung tätig.

\item Der Vorstand haftet nur für grobe Fahrlässigkeit.

\item Näheres regelt die Geschäftsordnung des Vorstandes.

\end{enumerate}

\paragr{Geschäftsführende Versammlung}
\begin{enumerate}
\item Die Geschäftsführende Versammlung ist die Versammlung der regulären Mitglieder. Sie wird vom geschäftsführenden Vorstand nach Bedarf oder auf schriftlichen Antrag von mindestens fünf oder ein Drittel der regulären Mitglieder (die jeweils kleinere Zahl ist maßgebend) einberufen. Die Einberufung erfolgt über das Protokoll der letzten Sitzung, die Webseite oder per Mail mindestens eine Woche vor dem Versammlungstermin.
\item Die Geschäftsführende Versammlung berät und beschließt alle Vereinsangelegenheiten, welche nicht ausdrücklich der Mitgliederversammlung oder dem Vorstand vorbehalten sind. Der Vorstand ist an Beschlüsse der geschäftsführenden Versammlung gebunden.
\item Eine Geschäftsführende Versammlung ist beschlussfähig, wenn sie ordnungsgemäß einberufen ist und wenn mindestens fünf oder ein Drittel der regulären Mitglieder (die jeweils kleinere Zahl ist maßgebend) teilnehmen. Die Teilnahme kann auch über Telekommunikationsdienste (z.B. Skype o.ä.) erfolgen.
\item Die geschäftsführende Versammlung fasst ihre Beschlüsse mit einfacher Mehrheit der an der Beschlussfassung teilnehmenden Stimmberechtigten. Die Stimmberechtigung ist auf die regulären Mitglieder beschränkt.
\end{enumerate}


\paragr{Teams}
\begin{enumerate}

\item Teams können von der Mitgliederversammlung und dem Vorstand zur Erledigung von mindestens einer bestimmten Aufgabe berufen werden.

\item Durch die Mitgliederversammlung sowie den Vorstand kann eine Änderung der bestimmten Aufgabe aufgetragen werden. Bei Änderung der Aufgabe sind Teammitglieder berechtigt das Team zu verlassen.

\item Teams sind gegenüber dem bestehenden Vorstand rechenschaftspflichtig.

\item Mitglieder von Teams können natürliche Personen, juristische Personen und Amtsträger qua officio werden.

\item Teammitglieder können unter Erklärung Ihres Einverständnisses durch die Mitgliederversammlung, den Vorstand oder das Team selbst berufen werden.

\item Zu der zugeordneten Aufgabe gesellen sich Folgende:
\begin{enumerate}
\item Benennung einer Ansprechperson für die restlichen Organe des Vereins.
\item Festlegung einer Geschäftsordnung.
\end{enumerate}

\item Mitglieder von Teams können sowohl auf eigenen Wunsch, durch Beschluss der Mitgliederversammlung, des Vorstandes oder des Teams abberufen werden.

\item Näheres regelt die Geschäftsordnung für Teams.

\end{enumerate}


\paragr{Kassenprüfung}
\begin{enumerate}

\item Die Mitgliederversammlung wählt für die Dauer von einem Jahr zwei Kassenprüfer/innen.

\item Diese/r darf nicht Mitglied des Vorstands sein.

\item Die Kassenprüfer/innen dürfen nicht Angehörige desselben Mitglieds gemäß \ref{mitglieder_regulaer} wie die Vorstände sein.

\item Wiederwahl ist zulässig.

\item Soll der Vorstand entlastet werden, ist der Mitgliederversammlung eine Woche vor deren Zusammentreten ein Kassenprüfbericht vorzulegen.

\end{enumerate}


\paragr{Datenschutz}

Näheres regelt die Datenschutzerklärung.


\paragr{Auflösung des Vereins}

Bei Auflösung oder Aufhebung des Vereins oder bei Wegfall steuerbegünstigter Zwecke fällt das Vermögen des Vereins an eine oder mehrere steuerbegünstigte Körperschaften zwecks Verwendung zur Förderung von Wissenschaft und Forschung, sowie Förderung der Erziehung, Volks- und Berufsbildung einschließlich der Studentenhilfe.


\end{enumerate}
\vspace{2cm}
Die Satzung wurde am 15.01.2020 im Rahmen der Mitgliederversammlung beschlossen.
	
\end{document}
